%% If you are using \orcid or academicons
%% icons, make sure you have the academicons
%% option here, and compile with XeLaTeX
%% or LuaLaTeX.
% \documentclass[10pt,a4paper,academicons]{altacv}

%% Use the "normalphoto" option if you want a normal photo instead of cropped to a circle
% \documentclass[10pt,a4paper,normalphoto]{altacv}

\documentclass[10pt,a4paper,ragged2e]{altacv}

%% AltaCV uses the fontawesome and academicon fonts
%% and packages.
%% See texdoc.net/pkg/fontawecome and http://texdoc.net/pkg/academicons for full list of symbols. You MUST compile with XeLaTeX or LuaLaTeX if you want to use academicons.

\usepackage{hyperref}

% Change the page layout if you need to
\geometry{left=1cm,right=9cm,marginparwidth=6.8cm,marginparsep=1.2cm,top=1.25cm,bottom=1.25cm}

% Change the font if you want to, depending on whether
% you're using pdflatex or xelatex/lualatex
\ifxetexorluatex
  % If using xelatex or lualatex:
    \setmainfont{Carlito}
\else
  % If using pdflatex:
    \usepackage[utf8]{inputenc}
    \usepackage[T1]{fontenc}
    \usepackage[default]{lato}
\fi

% Change the colours if you want to
\definecolor{VividPurple}{HTML}{3E0097}
\definecolor{SlateGrey}{HTML}{2E2E2E}
\definecolor{LightGrey}{HTML}{111111}

\definecolor{LightBlue}{HTML}{0070FF}
\definecolor{DarkBlue}{HTML}{333399}

\definecolor{Red}{HTML}{2980B9}

\colorlet{heading}{Red}
\colorlet{accent}{Red}
\colorlet{emphasis}{black}
\colorlet{body}{LightGrey}

% Change the bullets for itemize and rating marker
% for \cvskill if you want to
\renewcommand{\itemmarker}{{\small\textbullet}}
\renewcommand{\ratingmarker}{\faCircle}

%% sample.bib contains your publications
\addbibresource{sample.bib}

\begin{document}
\name{Felix Zhou}
\tagline{Mathemagician \& Engineer}{working with tranformation, analysis, and storage of structured and unstructured data}
\personalinfo{%
  % You can add your own with \printinfo{symbol}{detail}
    \email{felix990302@yahoo.ca}
    \phone{226-898-5226}
    \homepage{\href{http://zhou-felix.me}{zhou-felix.me}}
    \linkedin{\href{https://www.linkedin.com/in/felix-zhou}{felix-zhou}}
    \github{\href{https://www.github.com/felix990302}{felix990302}}
}

%% Make the header extend all the way to the right, if you want.
\begin{fullwidth}
    \makecvheader
\end{fullwidth}

%% Depending on your tastes, you may want to make fonts of itemize environments slightly smaller
\AtBeginEnvironment{itemize}{\small}


\cvsection[page1sidebar]{Experience}

\cvevent{Software Engineering Intern}{Encircle Inc.}{April 2018 -- August 2018}{Kitchener, ON}
\begin{itemize}
    \item Developed features in \textbf{Tornado} and \textbf{React} to improve Encircle's web app
    \item Introduced \textbf{WebSockets} and \textbf{Observables} to load and update data asynchonously in real-time, improving efficiency by \textbf{5x}
    \item Implemented email feature to send reports through \textbf{Model-Daemon Method}, removing need to exit app in user workflow
    \item Improved user experience with \textbf{functional} React Components, reducing re-rendering for smoother user experience 
\end{itemize}

\cvsection{Projects}

\cvevent{VM}{CS246E: Objected Oriented Programming (Advanced)}{November 2018}{\href{https://github.com/felix990302/vm}{github.com/felix990302/vm}}
\begin{itemize}
    \item Actualized a \textbf{C++14} clone of the command-line text editor \textit{vim}
    \item Followed \textbf{Object-Oriented Principles} and \textbf{Design Patterns} like \textbf{Decorator} and \textbf{Visitor} to produce modular and extensible code
    \item Implemented undos and redos through the \textbf{Command} pattern to minimize space complexity
\end{itemize}

\divider

\cvevent{Goose or Not}{Hack the North}{September 2018}{\href{https://github.com/felix990302/GooseOrNot}{github.com/felix990302/GooseOrNot}}
\begin{itemize}
    \item Created an \textbf{Online Learning} application which recognizes geese with \textbf{Flask} and \textbf{TypeScript-React}
    \item Designed a \textbf{PostgreSQL} database with \textbf{ORM} to improve extensibility
    \item Combined \textbf{NNLIB} and \textbf{XGBoost} for classification with \textbf{90\% accuracy}
    \item Deployed several \textbf{Heroku} instances with \textbf{PgBouncer} and \textbf{NGINX} for load balancing and increased transaction throughput
\end{itemize}
\textit{Inspired by Silicon Valley Season 4 Episode 4}

\divider

\cvevent{NNLib}{}{April 2018 -- September 2018}{\href{https://github.com/felix990302/nnlib}{github.com/felix990302/nnlib}}
\begin{itemize}
    \item Created a light-weight Neural Network library with only \textbf{NumPy} as dependency
    \item Featured \textbf{L1, L2 Regularization} and \textbf{Dropout} to combat variance
    \item Implemented \textbf{Back-Substitution} and \textbf{Gradient Descent} through Chain Rule
    \item Abided by \textbf{PyPI} package structure and conventions including full a \textbf{pytest} suite to improve compatibility with other Python packages
\end{itemize}

\divider

\cvevent{Faustin}{YHacks}{December 2017}{\href{https://github.com/tangsaidi/YHack}{github.com.tagsaidi/YHack}}
\begin{itemize}
    \item Provided real-time suggestions in reaction to emotions in vocal input
    \item Combined a \textbf{Deep Neural Network}, \textbf{IBM Watson} and \textbf{Volkaturi API} to analyze tonality and semantics to detect outliers due to factors such as sarcasm
    \item Classified speech in one of five emotional classes with \textbf{80\% Accuracy}
    \item Launched the application via \textbf{AWS} on a \textbf{CentOS} server
\end{itemize}


\end{document}
